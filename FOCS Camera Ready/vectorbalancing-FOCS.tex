
%% bare_conf.tex
%% V1.3
%% 2007/01/11
%% by Michael Shell
%% See:
%% http://www.michaelshell.org/
%% for current contact information.
%%
%% This is a skeleton file demonstrating the use of IEEEtran.cls
%% (requires IEEEtran.cls version 1.7 or later) with an IEEE conference paper.
%%
%% Support sites:
%% http://www.michaelshell.org/tex/ieeetran/
%% http://www.ctan.org/tex-archive/macros/latex/contrib/IEEEtran/
%% and
%% http://www.ieee.org/

%%*************************************************************************
%% Legal Notice:
%% This code is offered as-is without any warranty either expressed or
%% implied; without even the implied warranty of MERCHANTABILITY or
%% FITNESS FOR A PARTICULAR PURPOSE! 
%% User assumes all risk.
%% In no event shall IEEE or any contributor to this code be liable for
%% any damages or losses, including, but not limited to, incidental,
%% consequential, or any other damages, resulting from the use or misuse
%% of any information contained here.
%%
%% All comments are the opinions of their respective authors and are not
%% necessarily endorsed by the IEEE.
%%
%% This work is distributed under the LaTeX Project Public License (LPPL)
%% ( http://www.latex-project.org/ ) version 1.3, and may be freely used,
%% distributed and modified. A copy of the LPPL, version 1.3, is included
%% in the base LaTeX documentation of all distributions of LaTeX released
%% 2003/12/01 or later.
%% Retain all contribution notices and credits.
%% ** Modified files should be clearly indicated as such, including  **
%% ** renaming them and changing author support contact information. **
%%
%% File list of work: IEEEtran.cls, IEEEtran_HOWTO.pdf, bare_adv.tex,
%%                    bare_conf.tex, bare_jrnl.tex, bare_jrnl_compsoc.tex
%%*************************************************************************

% *** Authors should verify (and, if needed, correct) their LaTeX system  ***
% *** with the testflow diagnostic prior to trusting their LaTeX platform ***
% *** with production work. IEEE's font choices can trigger bugs that do  ***
% *** not appear when using other class files.                            ***
% The testflow support page is at:
% http://www.michaelshell.org/tex/testflow/



% Note that the a4paper option is mainly intended so that authors in
% countries using A4 can easily print to A4 and see how their papers will
% look in print - the typesetting of the document will not typically be
% affected with changes in paper size (but the bottom and side margins will).
% Use the testflow package mentioned above to verify correct handling of
% both paper sizes by the user's LaTeX system.
%
% Also note that the "draftcls" or "draftclsnofoot", not "draft", option
% should be used if it is desired that the figures are to be displayed in
% draft mode.
%
\documentclass[10pt, conference, compsocconf]{IEEEtran}
% Add the compsocconf option for Computer Society conferences.
%
% If IEEEtran.cls has not been installed into the LaTeX system files,
% manually specify the path to it like:
% \documentclass[conference]{../sty/IEEEtran}





% Some very useful LaTeX packages include:
% (uncomment the ones you want to load)


% *** MISC UTILITY PACKAGES ***
%
%\usepackage{ifpdf}
% Heiko Oberdiek's ifpdf.sty is very useful if you need conditional
% compilation based on whether the output is pdf or dvi.
% usage:
% \ifpdf
%   % pdf code
% \else
%   % dvi code
% \fi
% The latest version of ifpdf.sty can be obtained from:
% http://www.ctan.org/tex-archive/macros/latex/contrib/oberdiek/
% Also, note that IEEEtran.cls V1.7 and later provides a builtin
% \ifCLASSINFOpdf conditional that works the same way.
% When switching from latex to pdflatex and vice-versa, the compiler may
% have to be run twice to clear warning/error messages.






% *** CITATION PACKAGES ***
%
\usepackage{cite}
% cite.sty was written by Donald Arseneau
% V1.6 and later of IEEEtran pre-defines the format of the cite.sty package
% \cite{} output to follow that of IEEE. Loading the cite package will
% result in citation numbers being automatically sorted and properly
% "compressed/ranged". e.g., [1], [9], [2], [7], [5], [6] without using
% cite.sty will become [1], [2], [5]--[7], [9] using cite.sty. cite.sty's
% \cite will automatically add leading space, if needed. Use cite.sty's
% noadjust option (cite.sty V3.8 and later) if you want to turn this off.
% cite.sty is already installed on most LaTeX systems. Be sure and use
% version 4.0 (2003-05-27) and later if using hyperref.sty. cite.sty does
% not currently provide for hyperlinked citations.
% The latest version can be obtained at:
% http://www.ctan.org/tex-archive/macros/latex/contrib/cite/
% The documentation is contained in the cite.sty file itself.






% *** GRAPHICS RELATED PACKAGES ***
%
\ifCLASSINFOpdf
  % \usepackage[pdftex]{graphicx}
  % declare the path(s) where your graphic files are
  % \graphicspath{{../pdf/}{../jpeg/}}
  % and their extensions so you won't have to specify these with
  % every instance of \includegraphics
  % \DeclareGraphicsExtensions{.pdf,.jpeg,.png}
\else
  % or other class option (dvipsone, dvipdf, if not using dvips). graphicx
  % will default to the driver specified in the system graphics.cfg if no
  % driver is specified.
  % \usepackage[dvips]{graphicx}
  % declare the path(s) where your graphic files are
  % \graphicspath{{../eps/}}
  % and their extensions so you won't have to specify these with
  % every instance of \includegraphics
  % \DeclareGraphicsExtensions{.eps}
\fi
% graphicx was written by David Carlisle and Sebastian Rahtz. It is
% required if you want graphics, photos, etc. graphicx.sty is already
% installed on most LaTeX systems. The latest version and documentation can
% be obtained at: 
% http://www.ctan.org/tex-archive/macros/latex/required/graphics/
% Another good source of documentation is "Using Imported Graphics in
% LaTeX2e" by Keith Reckdahl which can be found as epslatex.ps or
% epslatex.pdf at: http://www.ctan.org/tex-archive/info/
%
% latex, and pdflatex in dvi mode, support graphics in encapsulated
% postscript (.eps) format. pdflatex in pdf mode supports graphics
% in .pdf, .jpeg, .png and .mps (metapost) formats. Users should ensure
% that all non-photo figures use a vector format (.eps, .pdf, .mps) and
% not a bitmapped formats (.jpeg, .png). IEEE frowns on bitmapped formats
% which can result in "jaggedy"/blurry rendering of lines and letters as
% well as large increases in file sizes.
%
% You can find documentation about the pdfTeX application at:
% http://www.tug.org/applications/pdftex





% *** MATH PACKAGES ***
%
\usepackage[cmex10]{amsmath}
\usepackage{amssymb,mathtools,amsthm}
% A popular package from the American Mathematical Society that provides
% many useful and powerful commands for dealing with mathematics. If using
% it, be sure to load this package with the cmex10 option to ensure that
% only type 1 fonts will utilized at all point sizes. Without this option,
% it is possible that some math symbols, particularly those within
% footnotes, will be rendered in bitmap form which will result in a
% document that can not be IEEE Xplore compliant!
%
% Also, note that the amsmath package sets \interdisplaylinepenalty to 10000
% thus preventing page breaks from occurring within multiline equations. Use:
%\interdisplaylinepenalty=2500
% after loading amsmath to restore such page breaks as IEEEtran.cls normally
% does. amsmath.sty is already installed on most LaTeX systems. The latest
% version and documentation can be obtained at:
% http://www.ctan.org/tex-archive/macros/latex/required/amslatex/math/





% *** SPECIALIZED LIST PACKAGES ***
%
%\usepackage{algorithmic}
% algorithmic.sty was written by Peter Williams and Rogerio Brito.
% This package provides an algorithmic environment fo describing algorithms.
% You can use the algorithmic environment in-text or within a figure
% environment to provide for a floating algorithm. Do NOT use the algorithm
% floating environment provided by algorithm.sty (by the same authors) or
% algorithm2e.sty (by Christophe Fiorio) as IEEE does not use dedicated
% algorithm float types and packages that provide these will not provide
% correct IEEE style captions. The latest version and documentation of
% algorithmic.sty can be obtained at:
% http://www.ctan.org/tex-archive/macros/latex/contrib/algorithms/
% There is also a support site at:
% http://algorithms.berlios.de/index.html
% Also of interest may be the (relatively newer and more customizable)
% algorithmicx.sty package by Szasz Janos:
% http://www.ctan.org/tex-archive/macros/latex/contrib/algorithmicx/




% *** ALIGNMENT PACKAGES ***
%
%\usepackage{array}
% Frank Mittelbach's and David Carlisle's array.sty patches and improves
% the standard LaTeX2e array and tabular environments to provide better
% appearance and additional user controls. As the default LaTeX2e table
% generation code is lacking to the point of almost being broken with
% respect to the quality of the end results, all users are strongly
% advised to use an enhanced (at the very least that provided by array.sty)
% set of table tools. array.sty is already installed on most systems. The
% latest version and documentation can be obtained at:
% http://www.ctan.org/tex-archive/macros/latex/required/tools/


%\usepackage{mdwmath}
%\usepackage{mdwtab}
% Also highly recommended is Mark Wooding's extremely powerful MDW tools,
% especially mdwmath.sty and mdwtab.sty which are used to format equations
% and tables, respectively. The MDWtools set is already installed on most
% LaTeX systems. The lastest version and documentation is available at:
% http://www.ctan.org/tex-archive/macros/latex/contrib/mdwtools/


% IEEEtran contains the IEEEeqnarray family of commands that can be used to
% generate multiline equations as well as matrices, tables, etc., of high
% quality.


%\usepackage{eqparbox}
% Also of notable interest is Scott Pakin's eqparbox package for creating
% (automatically sized) equal width boxes - aka "natural width parboxes".
% Available at:
% http://www.ctan.org/tex-archive/macros/latex/contrib/eqparbox/





% *** SUBFIGURE PACKAGES ***
%\usepackage[tight,footnotesize]{subfigure}
% subfigure.sty was written by Steven Douglas Cochran. This package makes it
% easy to put subfigures in your figures. e.g., "Figure 1a and 1b". For IEEE
% work, it is a good idea to load it with the tight package option to reduce
% the amount of white space around the subfigures. subfigure.sty is already
% installed on most LaTeX systems. The latest version and documentation can
% be obtained at:
% http://www.ctan.org/tex-archive/obsolete/macros/latex/contrib/subfigure/
% subfigure.sty has been superceeded by subfig.sty.



%\usepackage[caption=false]{caption}
%\usepackage[font=footnotesize]{subfig}
% subfig.sty, also written by Steven Douglas Cochran, is the modern
% replacement for subfigure.sty. However, subfig.sty requires and
% automatically loads Axel Sommerfeldt's caption.sty which will override
% IEEEtran.cls handling of captions and this will result in nonIEEE style
% figure/table captions. To prevent this problem, be sure and preload
% caption.sty with its "caption=false" package option. This is will preserve
% IEEEtran.cls handing of captions. Version 1.3 (2005/06/28) and later 
% (recommended due to many improvements over 1.2) of subfig.sty supports
% the caption=false option directly:
%\usepackage[caption=false,font=footnotesize]{subfig}
%
% The latest version and documentation can be obtained at:
% http://www.ctan.org/tex-archive/macros/latex/contrib/subfig/
% The latest version and documentation of caption.sty can be obtained at:
% http://www.ctan.org/tex-archive/macros/latex/contrib/caption/




% *** FLOAT PACKAGES ***
%
%\usepackage{fixltx2e}
% fixltx2e, the successor to the earlier fix2col.sty, was written by
% Frank Mittelbach and David Carlisle. This package corrects a few problems
% in the LaTeX2e kernel, the most notable of which is that in current
% LaTeX2e releases, the ordering of single and double column floats is not
% guaranteed to be preserved. Thus, an unpatched LaTeX2e can allow a
% single column figure to be placed prior to an earlier double column
% figure. The latest version and documentation can be found at:
% http://www.ctan.org/tex-archive/macros/latex/base/



%\usepackage{stfloats}
% stfloats.sty was written by Sigitas Tolusis. This package gives LaTeX2e
% the ability to do double column floats at the bottom of the page as well
% as the top. (e.g., "\begin{figure*}[!b]" is not normally possible in
% LaTeX2e). It also provides a command:
%\fnbelowfloat
% to enable the placement of footnotes below bottom floats (the standard
% LaTeX2e kernel puts them above bottom floats). This is an invasive package
% which rewrites many portions of the LaTeX2e float routines. It may not work
% with other packages that modify the LaTeX2e float routines. The latest
% version and documentation can be obtained at:
% http://www.ctan.org/tex-archive/macros/latex/contrib/sttools/
% Documentation is contained in the stfloats.sty comments as well as in the
% presfull.pdf file. Do not use the stfloats baselinefloat ability as IEEE
% does not allow \baselineskip to stretch. Authors submitting work to the
% IEEE should note that IEEE rarely uses double column equations and
% that authors should try to avoid such use. Do not be tempted to use the
% cuted.sty or midfloat.sty packages (also by Sigitas Tolusis) as IEEE does
% not format its papers in such ways.





% *** PDF, URL AND HYPERLINK PACKAGES ***
%
%\usepackage{url}
% url.sty was written by Donald Arseneau. It provides better support for
% handling and breaking URLs. url.sty is already installed on most LaTeX
% systems. The latest version can be obtained at:
% http://www.ctan.org/tex-archive/macros/latex/contrib/misc/
% Read the url.sty source comments for usage information. Basically,
% \url{my_url_here}.





% *** Do not adjust lengths that control margins, column widths, etc. ***
% *** Do not use packages that alter fonts (such as pslatex).         ***
% There should be no need to do such things with IEEEtran.cls V1.6 and later.
% (Unless specifically asked to do so by the journal or conference you plan
% to submit to, of course. )

\newtheorem{theorem}{Theorem}
\newtheorem{definition}[theorem]{Definition}
\newtheorem{conjecture}[theorem]{Conjecture}
\newtheorem{prop}[theorem]{Proposition}
\newtheorem{lemma}[theorem]{Lemma}
\newtheorem{corollary}[theorem]{Corollary}

\newcommand{\cut}[1]{}
\newcommand{\ceil}[1]{\lceil #1 \rceil}
\newcommand{\floor}[1]{\lfloor #1 \rfloor}

\newcommand{\R}{{\mathbb{R}}}
\newcommand{\C}{{\mathbb{C}}}
\newcommand{\Q}{{\mathbb{Q}}}
\newcommand{\N}{{\mathbb{N}}}
\renewcommand{\S}{\mathbb{S}}

\newcommand{\E}{\mathbb{E}}
\newcommand{\T}{\mathsf T}

\newcommand\eps{\varepsilon}
\newcommand{\eqdef}{:=}
\newcommand{\inner}[2]{\langle #1, #2 \rangle}
\newcommand{\set}[1]{\left\{ #1 \right\}}

\DeclareMathOperator{\vollb}{volLB}
\DeclareMathOperator{\detlb}{detLB}
\DeclareMathOperator{\disc}{disc}
\DeclareMathOperator{\hd}{hd}
\DeclareMathOperator{\vb}{vb}
\DeclareMathOperator{\ac}{ac}
\DeclareMathOperator{\vol}{vol}
\DeclareMathOperator{\cov}{cov}
\DeclareMathOperator{\conv}{conv}
\DeclareMathOperator{\lspan}{span}
\DeclareMathOperator{\speclb}{specLB}
\DeclareMathOperator{\tr}{tr}
\DeclareMathOperator{\bnds}{bounds}
\DeclareMathOperator{\rank}{rk}

\DeclarePairedDelimiter\norm{\lVert}{\rVert}



% correct bad hyphenation here
\hyphenation{op-tical net-works semi-conduc-tor}


\begin{document}
%
% paper title
% can use linebreaks \\ within to get better formatting as desired
\title{Balancing Vectors in Any Norm}


% author names and affiliations
% use a multiple column layout for up to two different
% affiliations

% \author{\IEEEauthorblockN{Authors Name/s per 1st Affiliation (Author)}
% \IEEEauthorblockA{line 1 (of Affiliation): dept. name of organization\\
% line 2: name of organization, acronyms acceptable\\
% line 3: City, Country\\
% line 4: Email: name@xyz.com}
% \and
% \IEEEauthorblockN{Authors Name/s per 2nd Affiliation (Author)}
% \IEEEauthorblockA{line 1 (of Affiliation): dept. name of organization\\
% line 2: name of organization, acronyms acceptable\\
% line 3: City, Country\\
% line 4: Email: name@xyz.com}
% }

% conference papers do not typically use \thanks and this command
% is locked out in conference mode. If really needed, such as for
% the acknowledgment of grants, issue a \IEEEoverridecommandlockouts
% after \documentclass

% for over three affiliations, or if they all won't fit within the width
% of the page, use this alternative format:
% 
%\author{\IEEEauthorblockN{Michael Shell\IEEEauthorrefmark{1},
%Homer Simpson\IEEEauthorrefmark{2},
%James Kirk\IEEEauthorrefmark{3}, 
%Montgomery Scott\IEEEauthorrefmark{3} and
%Eldon Tyrell\IEEEauthorrefmark{4}}
%\IEEEauthorblockA{\IEEEauthorrefmark{1}School of Electrical and Computer Engineering\\
%Georgia Institute of Technology,
%Atlanta, Georgia 30332--0250\\ Email: see http://www.michaelshell.org/contact.html}
%\IEEEauthorblockA{\IEEEauthorrefmark{2}Twentieth Century Fox, Springfield, USA\\
%Email: homer@thesimpsons.com}
%\IEEEauthorblockA{\IEEEauthorrefmark{3}Starfleet Academy, San Francisco, California 96678-2391\\
%Telephone: (800) 555--1212, Fax: (888) 555--1212}
%\IEEEauthorblockA{\IEEEauthorrefmark{4}Tyrell Inc., 123 Replicant Street, Los Angeles, California 90210--4321}}

\author{\IEEEauthorblockN{
    Daniel Dadush\IEEEauthorrefmark{1},
    Aleksandar Nikolov\IEEEauthorrefmark{2},
    Kunal Talwar\IEEEauthorrefmark{3}, 
    Nicole Tomczak-Jaegermann\IEEEauthorrefmark{4}}
  \IEEEauthorblockA{
    \IEEEauthorrefmark{1}
    Centrum Wiskunde \& Informatica,
    Amsterdam, Netherlands\\ 
    Email: dadush@cwi.nl}
  \IEEEauthorblockA{
    \IEEEauthorrefmark{2}
    %Department of Computer Science\\
    University of Toronto,
    Toronto, Ontario,Canada\\
    Email: anikolov@cs.toronto.edu}
  \IEEEauthorblockA{
    \IEEEauthorrefmark{3}
    Google Brain\\
    Email: kunal@google.com}
  \IEEEauthorblockA{
    \IEEEauthorrefmark{2}
    %Department of Mathematical and Statistical Sciences\\
    University of Alberta
    Edmonton, Alberta, Canada\\
    Email: nicole.tomczak@ualberta.ca}}

% \author{
%   \IEEEauthorblockN{Daniel Dadush}
%   \IEEEauthorblockA{
%     Centrum Wiskunde \& Informatica\\
%     Amsterdam, Netherlands\\ 
%     Email: dadush@cwi.nl}
%   \and
%   \IEEEauthorblockN{Aleksandar Nikolov}
%   \IEEEauthorblockA{
%     Department of Computer Science\\
%     University of Toronto\\
%     Toronto, Canada\\
%     Email: anikolov@cs.toronto.edu}
%   \and
%   \IEEEauthorblockN{Kunal Talwar}
%   \IEEEauthorblockA{
%     \IEEEauthorrefmark{3}
%     Google Brain\\
%     Email: kunal@google.com}
%   \and
%   \IEEEauthorblockN{Nicole Tomczak-Jaegermann}
%   \IEEEauthorblockA{
%     %Department of Mathematical and Statistical Sciences\\
%     University of Alberta\\
%     Edmonton, Canada\\
%     Email: nicole.tomczak@ualberta.ca}}




% use for special paper notices
%\IEEEspecialpapernotice{(Invited Paper)}




% make the title area
\maketitle


\begin{abstract}
In the vector balancing problem, we are given symmetric convex bodies $C$ and $K$ in $\R^n$, and our goal is to determine the minimum number $\beta \geq 0$, known as the vector balancing constant from $C$ to $K$, such that for any sequence of vectors in $C$ there always exists a signed combination of them lying inside $\beta K$. Many fundamental results in discrepancy theory, such as the Beck-Fiala theorem (Discrete Appl.~Math `81), Spencer's ``six standard deviations suffice'' theorem (Trans.~Amer.~Math.~Soc `85) and Banaszczyk's vector balancing theorem (Random Structures \& Algorithms `98) correspond to bounds on vector balancing
constants. 

The above theorems have inspired much research in recent years within theoretical computer science. In this work, we show that all vector balancing constants admit ``good'' approximate characterizations, with approximation factors depending only polylogarithmically on the dimension $n$. First, we show that a volumetric lower bound due to Banaszczyk is tight within a $O(\log n)$ factor. Our proof is algorithmic, and we show that Rothvoss's (FOCS `14) partial coloring algorithm can be analyzed to obtain these guarantees. Second, we present a novel convex program which encodes the ``best possible  way'' to apply Banaszczyk's vector balancing theorem for bounding vector balancing constants from above, and show that it is tight within an $O(\log^{2.5} n)$ factor. This also directly yields a corresponding polynomial time approximation algorithm both for vector balancing constants, and for the hereditary discrepancy of any sequence of vectors with respect to an arbitrary norm.    
  
%Our results yield the first guarantees which depend only polylogarithmically on the dimension of the norm ball $K$. All prior works required the norm to be polyhedral and incurred a dependence of $O(\sqrt{\log m})$, where $m$ is the number of facets. Our techniques rely on a novel combination of techniques from convex geometry and discrepancy theory. In particular, we give a new way to show lower bounds on Gaussian measures using only volumetric information, which may be of independent interest.
\end{abstract}

\begin{IEEEkeywords}
Discrepancy; Convex Geometry; Gaussian measure; M-ellipsoid; K-convexity.
\end{IEEEkeywords}


% For peer review papers, you can put extra information on the cover
% page as needed:
% \ifCLASSOPTIONpeerreview
% \begin{center} \bfseries EDICS Category: 3-BBND \end{center}
% \fi
%
% For peerreview papers, this IEEEtran command inserts a page break and
% creates the second title. It will be ignored for other modes.
\IEEEpeerreviewmaketitle


\section{Introduction}

The discrepancy of a set system is defined as the minimum, over the set of $\pm 1$ colorings of the elements, of the imbalance between the number of $+1$ and $-1$ elements in the most imbalanced set. Classical combinatorial discrepancy theory studies bounds on the discrepancy of set systems, in terms of their structure. 
%In classical combinatorial discrepancy, one studies for various set systems,
%over the set of $\pm 1$ colorings of the elements, what is the minimum
%worst-case imbalance, known as discrepancy, one guarantee between the number of
%$+1$ and $-1$ elements in every set? 
The tools developed for deriving bounds on the discrepancy of set
systems have found many applications in mathematics and computer
science~\cite{Matousek,Chazelle}, from the study of pseudorandomness,
to communication complexity, and most recently, to approximation
algorithms and privacy.  Here we study a geometric generalization of
combinatorial discrepancy, known as vector balancing, which captures
some of the most powerful techniques in the area, and is of intrinsic
interest. 

\paragraph{\bf Vector Balancing} In many instances, the best known techniques for finding
good bounds in combinatorial discrepancy were derived by working with more
general vector balancing problems, where convex geometric techniques can be
applied. Given symmetric convex bodies $C,K \subseteq \R^n$, the vector
balancing constant of $C$ into $K$ is defined as 
% i.e.~which induce the unit ball of a norm, and the
% goal is to understand the minimum number $M \geq 0$ such that for any sequence
% $(v_i)_{i=1}^N \in C$, there exists signs $(x_i)_{i=1}^N \in \set{-1,1}$ such
% that $\sum_{i=1}^N x_i v_i \in M D$. The minimum $M$ is denoted $\vb(C,D)$,
% the vector balancing constant of $C$ into $D$. 
\begin{multline*}
\vb(C, K) \eqdef \sup\Bigg\{ 
\min_{x \in \set{-1,1}^N} \Bigl\|\sum_{i = 1}^N x_i u_i \Bigr\|_{K}: \\N \in \mathbb{N}, u_1, \ldots, u_N \in C \Bigg\},
\end{multline*}
where $\norm{x}_K := \min \set{s \geq 0: x \in sK}$ is the norm induced by $K$.

As an example, one may consider Spencer's ``six standard deviations'' theorem~\cite{Spencer}, independently
obtained by Gluskin~\cite{gluskin}, which states that every set system on $n$
points and $n$ sets can be colored with discrepancy at most $O(\sqrt{n})$. In
the vector balancing context, the more general statement is that
$\vb(B_\infty^n,B_\infty^n) = O(\sqrt{n})$ (also proved
in~\cite{Spencer,gluskin}), where we use the notation $B_p^n = \set{x \in \R^n:
\norm{x}_p \leq 1}$, $p \in [1,\infty]$, to denote the unit ball of the $\ell_p$
norm. To encode Spencer's theorem, we simply represent the set system using its
incidence matrix $U \in \set{0,1}^{n \times n}$, where $U_{ji} = 1$ if element
$i$ is in set $j$ and $0$ otherwise. Here the columns of $U$ have $\ell_\infty$
norm $1$, and thus the sign vector $x \in \set{-1,1}^n$ satisfying $\norm{U
x}_\infty = O(\sqrt{n})$ indeed yields the desired coloring.   

In fact, vector balancing was studied earlier, and independently from
combinatorial discrepancy. In 1963 Dvoretzky posed the general problem
of determining $\vb(K, K)$ for a given symmetric convex body $K$. The
more general version with two different bodies was introduced by
Barany and Grinberg~\cite{baranygrinberg} who proved that for any
symmetric convex body $K$ in $\R^n$, $\vb(K, K) \le n$.  In addition
to Spencer's theorem, as described above, many other fundamental
discrepancy bounds, as well as conjectured bounds, can be stated in
terms of vector balancing constants. The Beck-Fiala theorem, which
bounds the discrepancy of any $t$-sparse set system (i.e.~one in which
each element appears in at most $t$ sets) by $2t-1$, can be recovered
from the bound $\vb(B_1^n,B_\infty^n) < 2$~\cite{beckfiala}. The
Beck-Fiala conjecture, which asks whether the bound for $t$-sparse set
systems can be improved to $O(\sqrt{t})$, is generalized by the
K{\'o}mlos conjecture~\cite{spencer-lectures}, which asks whether
$\vb(B_2^n,B_\infty^n) = O(1)$. One of the most important vector
balancing bounds is due to Banaszczyk~\cite{bana}, who proved that for
any convex body $K \subseteq \R^n$ of Gaussian measure $1/2$, one has
the bound $\vb(B_2^n,K) \leq 5$. In particular, this implies the bound
of $\vb(B_2^n,B_\infty^n) = O(\sqrt{\log n})$ for the K{\'o}mlos
conjecture.



\paragraph{\bf Hereditary Discrepancy.} While vector balancing gives useful
worst-case bounds, one is often interested in understanding the discrepancy
guarantees one can get for instances derived from a fixed set of vectors, known
as hereditary discrepancy. Given vectors $(u_i)_{i=1}^N$ in $\R^n$, the
discrepancy and hereditary discrepancy with respect to a symmetric convex body
$K \subseteq \R^n$ are defined as:
\begin{align*}
\disc((u_i)_{i = 1}^N,K) &\eqdef \min_{\eps_1, \ldots,\eps_N \in
  \{-1, 1\}}
\Bigl\|\sum_{i = 1}^N{\eps_i u_i}\Bigr\|_{K};\\
\hd((u_i)_{i = 1}^N, K) &\eqdef \max_{S \subseteq [N]}{\disc((u_i)_{i \in S}, K)}.
\end{align*}

When convenient, we will also use the notation $\hd(U,K) :=
\hd((u_i)_{i=1}^N,K)$, where $U := (u_1,\dots,u_N) \in \R^{n \times N}$, and
$\disc(U_S,K) := \disc((u_i)_{i \in S},K)$ for any subset $S \subseteq [N]$.  In
the context of set systems, $\ell_\infty$ hereditary discrepancy corresponds to
the worst-case discrepancy of any element induced subsystem, which gives a
robust notion of discrepancy, and can be seen as a measure of the
complexity of the set system. As an interesting example,
a set system has $\ell_\infty$ hereditary discrepancy $1$ if and only if its
incidence matrix is totally unimodular~\cite{GH-tum}. 

Beyond set systems, hereditary discrepancy can also usefully bound the
worst-case ``error'' required for rounding a fractional LP solution to an
integral one. More precisely, given any solution $y \in \R^n$ to a linear
programming relaxation $Ax \leq b$, $x \in [0,1]^n$, with $A \in \R^{m \times
n}, b \in \R^m$, of a binary IP, and given any norm $\norm{\cdot}$ on $\R^m$
measuring ``constraint violation'', one can ask what guarantees can be given on
$\min_{x \in \set{0,1}^n} \norm{A(y-x)}$? Using a well-known reduction of
Lov{\'a}sz, Spencer and Vesztergombi~\cite{LSV}, this error can be bounded by
$\hd(A,K)$ where $K$ is the unit ball of $\norm{\cdot}$. Furthermore,
this reduction guarantees that $x$ agrees with $y$ on its integer coordinates.
Note that we have the freedom to choose the norm $\norm{\cdot}$ so that the
error bounds meaningfully relate to the structure of the problem. Indeed, much
work has been done on the achievable ``error profiles'' one can obtain
algorithmically, e.g.~for which $\Delta \in \R^m_{> 0}$ we can always find $x
\in \set{0,1}^m$ satisfying $|A(y-x)| \leq \Delta$, $\forall y \in [0,1]^m$?
Note that the feasibility of an error profile can be recovered from a bound of
$1$ on the hereditary discrepancy with respect to the weighted $\ell_\infty$
norm $\norm{y-x}_{\Delta} = \max_{i \in [m]} |y_i-x_i|/\Delta_i$.  Indeed, in
many instances, this is (at least implicitly) how these bounds are proved. These
error profile bounds have been fruitfully leveraged for problems where small
``additive violations'' to the constraints are either allowed or can be repaired.
In particular, they were used in the recent $O(\log n)$-additive
approximation for bin packing~\cite{HobergRothvoss17}, an additive approximation
scheme for the train delivery problem~\cite{R12}, and additive approximations of
the degree bounded matroid basis problem~\cite{BN15}. 

\paragraph{\bf Discrepancy Minimization.} The original proofs of many of the
aforementioned discrepancy upper bounds were existential, and did not come with
efficient algorithms capable of constructing the requisite low discrepancy
colorings. %Over the last eight years, starting with the breakthrough work of
%Bansal~\cite{Bansal10}, who gave a constructive version of Spencer's theorem
%using random walk and semidefinite programming techniques, almost all known
%bounds have been made algorithmic. 
Starting with the breakthrough work of Bansal~\cite{Bansal10}, who gave a constructive version of Spencer's theorem using random walk and semidefinite programming techniques, nearly all known bounds have been made algorithmic in the last eight years.

One of the most important discrepancy minimization techniques is Beck's
partial coloring method, which covers most of the above discrepancy results
apart from Banaszczyk's vector balancing theorem. This method was first
primarily applied to $\ell_\infty$ discrepancy minimization problems of the form
\[
\min_{x \in \set{-1,1}^n} \Big\|\sum_{i=1}^n x_i v_i\Big\|_\infty, \text{ where
} (v_i)_{i=1}^n \in \R^m.
\]
As before, the goal is not to solve such problems near-optimally but
instead to find solutions satisfying a guaranteed error bound. The
partial coloring method solves this problem in phases, where at each
phase it ``colors'' (i.e.~sets to $\pm 1$) at least a constant
fraction of the remaining uncolored variables.  This yields $O(\log
n)$ partial coloring phases, where the discrepancy of the full
coloring is generally bounded by the sum of discrepancies incurred in
each phase. The existence of low discrepancy partial colorings was
initially established via the pigeon hole principle and arguments
based on the probabilistic and the entropy methods. In particular, the
entropy method gave a general sufficient condition for the feasibility
of any error profile (as above) with respect to partial
colorings. This method was made constructive by Lovett and
Meka~\cite{lovettmeka} using random walk techniques. The partial
coloring method was generalized by Giannopoulos~\cite{giann} to the
general vector balancing setting using Gaussian measure. Precisely, he
showed that if a symmetric convex body $K \subseteq \R^n$ has Gaussian
measure at least $2^{-c n}$, for $c$ small enough, then for any
sequence of vectors $v_1,\dots,v_n \in B_2^n$, there exists a partial
coloring $x \in \set{-1,0,1}^n$, having support at least $n/2$, such
that $\sum_{i=1}^n x_i v_i \in O(1) K$. This method was made
constructive by Rothvoss~\cite{rothvoss-giann}, using a random
projection algorithm, and later by Eldan and Singh~\cite{ES14} who
used the solution of a random linear maximization problem. An
important difference between the constructive and existential partial
coloring methods, is that the constructive methods only guarantee that
the ``uncolored'' coordinates of a partial coloring $x$ are in
$(-1,1)$ instead of equal to $0$. This relaxation seems to make the
constructive methods more robust, i.e.~the conditions needed for such
``fractional'' partial colorings are somewhat milder, without having
noticeable drawbacks in most applications.

The main alternative to the partial coloring method comes from Banaszczyk's
vector balancing theorem~\cite{bana}. 
 Banaszczyk's method proves the existence of a full coloring when
$K$ has gaussian measure $1/2$, in contrast to Giannopoulos's result
which gives a partial coloring but requires measure only $2^{-cn}$.
Banaszczyk's method was only very recently made constructive in
the sequence of works~\cite{BDG16,DGLN16,BDGL18}. In particular,~\cite{DGLN16}
showed an equivalence of Banaszczyk's theorem to the existence of certain
subgaussian signing distributions, and~\cite{BDGL18} gave a random walk-based
algorithm to build such distributions.

\subsection{Approximating Vector Balancing and Hereditary Discrepancy}
Given the powerful tools mentioned above, a natural question is
whether they can be extended to get nearly optimal bounds for any vector
balancing or hereditary discrepancy problem. More precisely, we will be
interested in the following computational and mathematical questions: 

\begin{enumerate}
\item Given vectors $(u_i)_{i=1}^N$ and a symmetric convex body $K$ in $\R^n$,
can we (a) efficently compute a coloring whose $K$-discrepancy is approximately
bounded by $\hd((u_i)_{i=1}^N,K)$? (b) efficiently approximate $\hd((u_i)_{i=1}^N,K)$?
\item Given two symmetric convex bodies $C,K \subseteq \R^n$, does $\vb(C,K)$
admit a ``good'' characterization? Namely, are there simple certificates which
certify nearly tight upper and lower bounds on $\vb(C,K)$?
\end{enumerate}

To begin, a few remarks are in order.  Firstly, question 2 can be inefficiently
encoded as question 1b, by letting $(u_i)_{i=1}^N$ denote a sufficiently fine
net of $C$. Thus ``good'' characterizations for hereditary discrepancy transfer
over to vector balancing, and, for this reason, we restrict for now the discussion to the
former. For question 1a, one may be tempted to ask whether we can directly
compute a coloring whose $K$-discrepancy is approximately
$\disc((u_i)_{i=1}^N,K)$ instead of $\hd((u_i)_{i=1}^N,K)$. Unfortunately, even
for $K = B_\infty^n$ and $(u_i)_{i=1}^n \in [-1,1]^n$, it was shown in
\cite{CNN11} that it is NP-hard to
distinguish whether $\disc((u_i)_{i=1}^n,B_\infty^n)$ is $0$ or
$\Omega(\sqrt{n})$ (note that $O(\sqrt{n})$ is guaranteed by Spencer's theorem),
thus one cannot hope for any non-trivial approximation guarantee in this
context. 

We now discuss prior work on these questions and then continue with our
main results. 

\subsection{Prior work.} For both questions 1 and and 2 above, prior
work has mostly dealt with the case of $\ell_\infty$ or $\ell_2$
discrepancy. Bounds on vector balancing constants from $\ell_p$ to
$\ell_q$ for some $p$ and $q$ have also been studied, as
described earlier, but without a unified approach. The question of
obtaining near-optimal results for general vector balancing and
hereditary discrepancy problems has not been studied
before.

In terms of coloring algorithms, Bansal~\cite{Bansal10} gave a partial coloring
based random walk algorithm which on $U \in \R^{m \times n}$, produces a full
coloring with $\ell_\infty$ discrepancy $O(\sqrt{\log m \log \rank(U)}
\hd(U,B_\infty^m))$, where $\rank(U)$ is the rank of $U$. Recently,
Larsen~\cite{Larsen17} gave an algorithm for the $\ell_2$ norm achieving
discrepancy $O(\sqrt{\log(\rank(U))}\hd(U,B_2^m))$. 

In terms of certifying lower bounds on $\hd(U,B_\infty^m))$, the main tool has
been the so-called determinant lower bound of~\cite{LSV}, where it was shown that
\[
\hd(U,B_\infty^m) \geq \detlb(U) := \max_k \max_B \frac{1}{2}|\det(B)|^{1/k}
\]       
with the maximum over $k \times k$ submatrices $B$ of $U$.
Matou\v{s}ek~\cite{Matousek11} built upon the results of~\cite{Bansal10} to show that 
\[
\hd(U,B_\infty^m) \leq O(\sqrt{\log m} \log\rank(U) \detlb(U)).
\]
For certifying tight upper bounds,~\cite{NT15,disc-gamma2} showed that the
$\gamma_2$ norm of $U$, defined by 
\begin{multline*}
\gamma_2(U) := \min \{\|A\|_{2 \rightarrow \infty} \|B\|_{1 \rightarrow 2}: U
= A B, \\A \in \R^{m \times k}, B \in \R^{k \times n}, k \in \N\}  
\end{multline*}
where $\|A\|_{2 \rightarrow \infty}$ is the maximum $\ell_2$ norm of any row of $A$,
and $\|B\|_{1 \rightarrow 2}$ is the maximum $\ell_2$ norm of any column of $B$,
satisfies
\begin{align*}
\Omega(\gamma_2(U)/\log(\rank(U))) &\leq \detlb(U) \\
&\leq \hd(U,B_\infty^m) \leq O(\sqrt{\log m} \gamma_2(U)) 
\end{align*}
which implies a $O(\sqrt{\log m} \log \rank(U))$ approximation to $\ell_\infty$
hereditary discrepancy. For $\ell_2$, it was shown in~\cite{NT15}
that a relaxation of $\gamma_2$ yields an $O(\log \rank(U))$-approximation to
$\hd(U,B_2^m)$. We note that part of the strategy of~\cite{NT15,disc-gamma2}
is to replace the $\ell_\infty$ norm via an averaged version of $\ell_2$, where
one optimizes over the averaging coefficients, which makes the $\ell_2$ norm by
itself an easier special case.

\paragraph{\bf Moving to general norms.} While at first glance it may seem that
the above techniques for $\ell_\infty$ do not apply to more general norms, this
is in some sense deceptive. Notwithstanding complexity considerations, every
norm can be isometrically embedded into $\ell_\infty$, where in particular any
polyhedral norm with $m$ facets can be embedded into $B^m_\infty$. Vice versa,
starting from a matrix $U \in \R^{m \times N}$, with $\rank(U) = n$ and rank
factorization $U = A B$, it is direct to verify that $\hd(U,B^m_\infty) = \hd(B,K)$,
where $K = \set{x \in \R^n: \|Ax\|_\infty \leq 1}$ is an $n$-dimensional symmetric
polytope with $m$ facets. Thus, for any $U \in \R^{n \times N}$, one can
equivalently restate the guarantees of~\cite{Bansal10} as yielding colorings of
discrepancy $O(\sqrt{\log m \log n} \hd(U,K))$ and of~\cite{disc-gamma2} as a
$O(\sqrt{\log m}\log n)$ approximation to $\hd(U,K)$ for any $n$-dimensional
symmetric polytope $K$ with $m$ facets. A natural question is therefore
whether there exist corresponding coloring and approximation algorithms whose
guarantees depend only polylogarithmically on the dimension of the norm and not on
the complexity of its representation. The upper and lower bound tools
mentioned above (the determinant lower bound and the $\gamma_2$ norm,
respectively) are insufficient for this task, as the polylogarithmic
dependence on $m$ in the bounds is known to be inherent. 

We note that polynomial bounds in $n$ for general $K$ can be achieved by simply
approximating $K$ by a sandwiching ellipsoid $E \subseteq K \subseteq \sqrt{n}
E$ and applying the corresponding results for $\ell_2$, which yield $O(\sqrt{n
\log n})$ coloring and $O(\sqrt{n}\log n)$ approximations guarantees
respectively. Interestingly, these guarantees are identical to what can be
achieved by replacing $K$ by a symmetric polytope with $3^n$ facets, giving a sandwiching factor of $2$, and applying the $\ell_\infty$ results.

\section{Results} 
Our main results are that polylogarithmic approximations to hereditary
discrepancy and vector balancing constants with respect to arbitrary
norms are indeed possible. In particular, given $U \in \R^{n \times
  N}$ and a symmetric convex body $K \subseteq \R^n$ (specified by an
appropriate oracle), we give randomized polynomial time algorithms for
computing colorings of discrepancy $O(\log n \hd(U,K))$ and
approximating $\hd(U,K)$ up to an $O(\log^{2.5} n)$ factor.  Furthermore,
if $K$ is a polyhedral norm with at most $m$ facets, our approximation
algorithm for $\hd(U,K)$ always achieves a tighter approximation
factor than the $\gamma_2$ bound, and hence gives an $O(\min \set{\log
  n \sqrt{\log m}, \log^{2.5} n})$ approximation. To achieve these
results, we first show that Rothvoss' partial coloring
algorithm~\cite{rothvoss-giann} is nearly optimal for general
hereditary discrepancy by showing near-tightness with respect to a
volumetric lower bound of Banaszczyk~\cite{Bana93}. Second, we show
that the ``best possible way'' to apply Banaszczyk's vector balancing
theorem~\cite{bana} for the purpose of bounding $\hd(U,K)$ from above can
be encoded as a convex program, and prove that this bound is tight to
within an $O(\log^{2.5} n)$ factor. As a consequence, we show that
Banaszczyk's theorem is essentially ``universal'' for vector
balancing. To analyze these approaches we rely on a novel combination
of tools from convex geometry and discrepancy. In particular, we give
a new way to prove lower bounds on Gaussian measure using only
volumetric information, which could be of independent interest.
Furthermore, we make a natural geometric conjecture which would imply
that Rothvoss' algorithm is (in a hereditary sense) optimal for
finding partial colorings in any norm, and prove the conjecture for
the special case of $\ell_2$.

Comparing to prior work, our coloring and hereditary discrepancy
approximation algorithms give uniformly better (or at at least no worse)
guarantees in almost every setting which has been studied. Furthermore our
methods provide a unified approach for studying discrepancy in arbitrary norms,
which we expect to have further applications.

Interestingly, our results imply a tighter relationship between vector balancing
and hereditary discrepancy than one might initially expect. We observe
that neither the volumetric lower bound we use nor our factorization based upper
bound ``see'' the difference between vector balancing constants and
hereditary discrepancy. More precisely, both bounds remain
invariant when replacing $\hd(U,K)$ by $\vb(\conv\set{\pm u_i: i \in [N]},K)$. This
has the relatively non-obvious implication that 
\begin{multline}
\label{eq:hd-to-vb}
\hd(U,K) \leq \vb(\conv\set{\pm u_i: i \in [N]},K) \\\leq O(\log n) \hd(U,K).
\end{multline}
We believe it is an interesting question to understand whether a polylogarithmic
separation indeed exists between the above quantities (we are currently unaware
of any examples), as it would give a tangible geometric obstruction for tighter
approximations.    

\section{Outline of the Proofs}

Starting with hereditary discrepancy, to push beyond the limitations of prior
approaches the first two tasks at hand are: (1) find a stronger lower bound and
(2) develop techniques to avoid the ``union bound''. Fortunately, a solution to the
first problem was already given by Banaszczyk\cite{Bana93}, which we present in
slightly adapted form below.

\begin{lemma}[Volume Lower Bound]%{lemma}{vollbstat} 
\label{lem:vol-lb}
Let $U=(u_1,\dots,u_N) \in \R^{n \times N}$ and $K \subseteq \R^n$ be a
symmetric convex body. For $S \subseteq [N]$, let $U_S$ denote the
submatrix of $U$ consisting of the columns
indexed by $S$. For $k \in [n]$, define 
\begin{multline}
\label{eq:vol-lb}
\vollb^{\rm h}_k((u_i)_{i=1}^N,K) \eqdef \vollb^{\rm h}_k(U, K) \\
\eqdef
\max_{S \subseteq [N],|S|=k} \vol_k(\{x \in \R^k: U_S x \in K\})^{-1/k}.
\end{multline}
Then, we have that
\begin{multline}
\label{eq:vol-lbh}
\vollb^{\rm h}((u_i)_{i=1}^N,K) \eqdef \vollb^{\rm h}(U,K) \\
\eqdef \max_{k \in [n]} \vollb^{\rm h}_k(U,K) \leq \hd(U,K). 
\end{multline}
\end{lemma}

A formal proof of the above is given in the full version, and follows
from the argument in \cite{Bana93}. At a high level, the proof is a simple covering
argument, where it is argued that for any subset $S$, $|S|=k$, every point in
$[0,1]^k$ is at distance at most $\hd(U,K)$ from $\set{0,1}^k$ under the norm
induced by $C := \set{x \in \R^k: U_S x \in K}$. Equivalently an $\hd(U,K)$
scaling of $C$ placed around the points of $\set{0,1}^k$ cover $[0,1]^k$, and
hence by a standard lattice argument must have volume at least that of
$[0,1]^k$, namely $1$. This yields the desired lower bound after rearranging. 

We note that the volume lower bound extends in the obvious way to vector
balancing. In particular, for two symmetric convex bodies $C,K
\subseteq \R^n$, we define
\begin{equation*}
%\label{eq:vol-lb-vb}
\vollb^{\rm h}(C,K) \eqdef \sup \vollb((u_i)_{i=1}^k,K),
\end{equation*}
with the supremum over $k \in [n]$ and sequences $u_1,\dots,u_k \in
C$. Then we have
\begin{equation}
\label{eq:vol-lb-vb}
\vollb^{\rm h}(C,K)  \geq \vb(C,K).
\end{equation}
This lower bound can be substantially stronger than the determinant lower
bound for $\ell_\infty$ discrepancy. As a simple example, let $U \in \R^{2^n
\times n}$ be the matrix having a row for each vector in $\set{-1,1}^n$. Since
$U$ has rank $n$, the determinant lower bound is restricted to $k \times k$
matrices for $k \in [n]$. Hadamard's inequality implies that for any $k \times k$
matrix $B$ with $\pm 1$ entries we have $|\det(B)|^{1/k} \leq \sqrt{k} \leq
\sqrt{n}$. A moment's thought, however, reveals that for $x \in \R^n$,
$\norm{Ux}_\infty = \norm{x}_1$ and hence any coloring $x \in \set{-1,1}$ must
have discrepancy $\norm{x}_1 = n$. Using the previous logic, the volume lower
bound yields by standard estimates
\begin{align*}
\vollb(U,B^m_\infty) &\geq \vol_n(\set{x \in \R^n: \norm{x}_1 \leq 1})^{-1/n}\\
&= \vol_n(B_1^n)^{-1/n} = (n!/2^n)^{1/n} \geq n/(2e), 
\end{align*}
which is essentially tight. 

\subsection{From Volume to Coloring} 

The above example gives hope that the volume lower bound can
circumvent a dependency on the facet complexity of the norm. Our first
main result shows that indeed this is the case:

\begin{theorem}[Tightness of the Volume Lower Bound]%{theorem}{tightvollb} 
\label{thm:tightness}
For any $U \in \R^{n \times N}$ and symmetric convex body $K$ in $\R^n$, we have that
\begin{equation*}
\label{eq:tightness}
\hspace{-0.8em}\vollb^{\rm h}(U,K) \leq \hd(U,K) \leq O(\log(n) \vollb^{\rm h}(U,K)).
\end{equation*}
Furthermore, there exists a randomized polynomial time algorithm that computes a
coloring of $U$ with $K$-discrepancy $O(\log n \vollb^{\rm h}(U,K))$, given a
membership oracle for $K$. 
\end{theorem}

We note that the above immediately implies the corresponding approximate
tightness of the volume lower bound for vector balancing. The above bound can
also be shown to be tight. In particular, the counterexample to the
3-permutations conjecture from~\cite{NNN12}, which has $\ell_\infty$ discrepancy
$\Omega(\log n)$, can be shown to have volume lower bound $O(1)$. The
computations for this are somewhat technical, so we defer a detailed discussion
to the full version. As mentioned previously, an interesting property of the
volume lower bound is its invariance under taking convex hulls, namely $\vollb^{\rm
h}(\conv\set{\pm U},K) = \vollb^{\rm h}(U,K)$. In combination with
Theorem~\ref{thm:tightness}, this establishes the claimed
inequality~\eqref{eq:hd-to-vb}. This invariance is proved in
the full version by showing that  the volume lower bound is convex in
each $u_i$, and hence maximized at extreme points.

Our proof of Theorem~\ref{thm:tightness} is algorithmic, and relies on iterated
applications of Rothvoss's partial coloring algorithm. We now explain our high
level strategy as well as the differences with respect to prior approaches. 

For simplicity of the presentation, we shall assume that $U=(e_1,\dots,e_n) \in
\R^{n \times n}$ and that the volume lower bound $\vollb^{\rm
h}((e_i)_{i=1}^n,K)=1$. This can be (approximately) achieved by applying a
standard reduction to the case where $U$ is non-singular, so $N \leq n$,
``folding'' $U$ into $K$, and appropriately guessing the volume lower
bound. 

For any subset $S \subseteq [n]$, let $K_S := \set{x \in K: x_i = 0~\forall i \in [n]
\setminus S}$ denote the coordinate section of $K$ induced by $S$. Since the
vectors of $U$ now correspond to the coordinate basis, it is direct to verify that
\[
\vollb^{\rm h}((e_i)_{i=1}^n,K) = \max_{S \subseteq [n],k:=|S|}
\vol_k(K_S)^{-1/k} .
\]  
In particular, the assumption $\vollb^{\rm h}((e_i)_{i=1}^n,K) = 1$ implies that
\begin{equation}
\label{eq:big-volume}
\vol_{|S|}(K_S) \geq 1,~\forall S \subseteq [n]. 
\end{equation}
Under this condition, our goal can now be stated as finding a coloring $x \in
\set{-1,1}^n \in O(\log n) K$.

When $K$ is a symmetric polytope \cut{$|Ax| \leq 1$,} with $m$ facets,
the algorithm by Bansal~\cite{Bansal10} uses a ``sticky'' random walk
inside $[0,1]^n$ with increments computed via an SDP. The SDP is used
to guarantee that the variance in the normal direction of any facet is
at bounded by $s\hd((e_i)_{i=1}^n,K)^2$, while the variance in the
direction of any (active) coordinate directions is at least $s$, for a
small parameter $s$. As this only gives probabilistic error
guarantees for each constraint in isolation, a union bound is used to
get a global guarantee, incurring the $O(\sqrt{\log m})$ dependence in
the final bound.

To avoid the ``union bound'', we instead use Rothvoss's partial coloring
algorithm, which simply samples a random Gaussian vector $X \in \R^n$ and
computes the closest point in Euclidean distance $x$ to $X$ in $K \cap [-1,1]^n$
as the candidate partial coloring. As long as $K$ has ``large enough'' Gaussian
measure, Rothvoss shows that $x$ has at least
a constant fraction of its components at $\pm 1$. While this method can in
 better leverage the geometry of $K$ than Bansal's method (in particular,
it does not need an explicit description of $K$), it is apriori unclear why Gaussian
measure should be large enough in the present context. 

Our main technical result is that if all the coordinate sections of
$K$ have volume at least $1$ (i.e.~condition~\eqref{eq:big-volume}),
then there indeed exists a section of $K$ of dimension close to $n$,
whose Gaussian measure is ``large'' after an appropriate
scaling. Specifically, we show that for any $\delta \in (0,1)$, there
exists a subspace $H$ of dimension $(1-\delta)n$ such that the
Gaussian measure of $2^{O(1/\delta)} (K \cap H)$ is at least
$2^{-\delta n}$ (the exact statement is given in the full version). We
sketch the ideas in the next subsection.

The existence of a large section of $K$ with Gaussian measure which is
not too small in fact suffices to run Rothvoss's partial coloring
algorithm. Conveniently, one does not need to know the section
explicitly, as its existence is only used in the analysis of the
algorithm. Since condition~\ref{eq:big-volume} is hereditary, we can
now find partial colorings of $K$-discrepancy $O(1)$ on any subset of
coordinates.  Thus, applying $O(\log n)$ partial coloring phases in
the standard way yields the desired full coloring.

A useful restatement of the above is that Rothvoss's algorithm can
always find partial colorings with discrepancy $O(1)$ times the volume
lower bound. 
%We note that this guarantee is a natural by-product of
%the algorithm (once one has guessed the appropriate scaling), which
%does not need to be explicitly enforced as in Bansal's algorithm.

\subsection{Finding a section with large Gaussian measure.} We now sketch how
to find a section of $K$ of large Gaussian measure under the assumption that
$\vol_{|S|}(K_S) \geq 1, \forall S \subseteq [n]$. The main tool we require is
the M-ellipsoid from convex geometry~\cite{Milman86-reverseBM}. The M-ellipsoid
$E$ of $K$ is an ellipsoid which approximates $K$ well from the perspective of
covering, that is $2^{O(n)}$ translates of $E$ suffice to cover $K$ and vice
versa. 

The main idea is to use the volumetric assumption to show that the longest
$(1-\delta)n$ axes of $E$, for $\delta \in (0,1)$ of our choice, have length at
least $\sqrt{n}2^{-O(1/\delta)}$, and then use the subspace generated by these
axes for the section of $K$ we use. On this subspace $H$, we have that a
$2^{O(1/\delta)}$ scaling of $E \cap H$ contains the $\sqrt{n}$ ball, and thus
by the covering estimate $2^{O(n)}$ translates of $2^{O(1/\delta)}(K \cap H)$
cover the $\sqrt{n}$ ball. Since the $\sqrt{n}$ ball on $H$ has Gaussian
measure at least $1/2$, the prior covering estimate indeed implies that
$2^{O(1/\delta)}(K \cap H)$ has Gaussian measure $2^{-O(n)}$, noting that
shifting $2^{O(1/\delta)}(K \cap H)$ away from the origin only reduces Gaussian
measure. Using an M-ellipsoid with appropriate regularity properties
(as in\cite{Pisier89}), one can scale $K \cap H$ by another $2^{O(1/\delta)}$
factor so that the preceding argument yields Gaussian measure at least
$2^{-\delta n}$. 

We now explain why the $(1-\delta)n$ longest axes of $E$ are indeed
long enough. By the covering estimates, for any $S \subseteq [n]$,
$|S| = \delta n$, the sections $E_S$ and $K_S$ satisfy
\[
\vol_{\delta n}(E_S)^{1/{\delta n}} \geq 2^{-O(1/\delta)} \vol_{\delta
n}(K_S)^{1/{\delta n}} \geq 2^{-O(1/\delta)},
\]
where the last inequality is by assumption. Using a form of the
restricted invertibility principle for determinants (see e.g.~the full
version of\cite{N15}), one can show that if all coordinate sections of
$E$ of dimension $\delta n$ have large volume, then so does every
section of $E$ of the same dimension.  Precisely, one gets that
\begin{multline*}
\min_{\dim(W)=\delta n} \vol_{\delta n}(E \cap W)^{1/\delta n} \geq\\ 
\binom{n}{\delta n}^{-1/\delta n} \min_{|S|=\delta n} \vol_{\delta
  n}(E_S)^{1/\delta n}\geq 2^{O(-1/\delta)}.
\end{multline*}
In particular, the above implies that the geometric average of the
lengths of the \emph{shortest} $\delta n$ axes of $E$ (corresponding
to the minimum volume section of $E$), must be at least
$\sqrt{n}2^{-O(1/\delta)}$ since the ball of volume $1$ in dimension
$\delta n$ has radius $\Omega(\sqrt{\delta n})$. But then, the longest
$(1-\delta) n$ axes all have have length
$\sqrt{n}2^{-O(1/\delta)}$. This completes the proof sketch.

\subsection{ The Discrepancy of Partial Colorings} Our analysis of Rothvoss's
algorithm opens up the tantalizing possibility that it may indeed be optimal for
finding partial colorings in a hereditary sense. More precisely, we conjecture
that if when run on an instance $U$ with norm ball $K$, the algorithm almost
always produces partial colorings with $K$-discrepancy at least $D$, then there
exists a subset of $S$ of the columns of $U$ such that every partial coloring
of $U_S$ has discrepancy $\Omega(D)$. The starting point for this conjecture is
our upper bound of $O(\vollb^{\rm h}(U,K))$, on the discrepancy of the partial
colorings the algorithm computes. It remains to show that the volume
lower bound is also a lower bound on the hereditary discrepancy of
\emph{partial} colorings. We now provide a purely geometric conjecture,
which would imply the above ``hereditary optimality'' for Rothvoss's algorithm. 

As in the last subsection, we may assume that $U=(e_1,\dots,e_n)$ is the
standard basis of $\R^n$ and that $\vollb((e_i)_{i=1}^n,K) = 1$. To prove the
conjecture, it suffices to show that there exists some subset $S \subseteq [n]$ of
coordinates, such that all partial colorings have $K$-discrepancy $\Omega(1)$.
For concreteness, let us ask for partial colorings which color at least $|S|/2$
coordinates (the precise constant will not matter). For $x \in [-1,1]^n$, define
$\bnds(x) = \set{i \in [n]: x_i \in \set{-1,1}}$.  With this notation, our goal
is to find $S \subseteq [n]$, such that $\forall x \in [-1,1]^S$, $|\bnds(x)|
\geq |S|/2$, $\norm{\sum_{i \in S} x_i e_i}_K \geq \Omega(1)$.

We explain the candidate geometric obstruction to low discrepancy partial
colorings, which is a natural generalization of the so-called spectral lower
bound for $\ell_2$ discrepancy. Assume that for some subset $S \subseteq
[n]$, we have that 
\begin{equation}
\label{eq:spec-obst-intro}
K_S \subseteq c \sqrt{|S|} B_2^S,
\end{equation}
where $B_2^S := (B_2^n)_S$, for some constant $c > 0$. Since any partial
coloring $x \in [-1,1]^S$, $|\bnds(x)|\geq |S|/2$, clearly has $\norm{x}_2 \geq
\sqrt{|S|/2}$, we must have that
\begin{equation}
\label{eq:spec-lb-intro}
\frac{1}{c\sqrt{2}} 
\leq \Bigl\|\sum_{i \in S} x_i e_i\Bigr\|_{c\sqrt{|S|}B_2^S} 
\leq \Bigl\|\sum_{i \in S} x_i e_i\Bigr\|_{K_S}.
\end{equation}
In particular, every partial coloring on $S$ has discrepancy at least $\tfrac{1}{c
\sqrt{2}} = \Omega(1)$, as desired.

Given the above, we may now reduce the conjecture to the following natural
geometric question:

\begin{conjecture}[Restricted Invertibility for Convex Bodies]%{conjecture}{ripconvex} 
\label{conj:rip-convex}
There exists an absolute constant $c \geq 1$, such that for any $n \in \N$
and symmetric convex body $K \subseteq \R^n$ of volume at most $1$, there exists $S
\subseteq [n]$, $S \neq \emptyset$, such that $K_S \subseteq c\sqrt{|S|}B_2^S$. 
\end{conjecture}

To see that this indeed implies the required statement, note that if
$\vollb((e_i)_{i=1}^n,K) = 1$, then by definition there exists $A \subseteq
[n]$, $|A| \geq 1$, such that $\vol_{|A|}(K_A) \leq 1$. Now applying the above
conjecture to $K_A$ yields the desired result. 

Two natural relaxations of the conjectures are to ask (1) does it hold for
ellipsoids and (2) does it hold for general sections instead of coordinate
sections? Our main evidence for this conjecture is that indeed both these
statements are true. We note that (1) indeed implies the optimality of
Rothvoss's partial coloring algorithm for $\ell_2$ discrepancy. Our results here
are slightly stronger than (1)+(2), as we in some sense manage to get ``halfway
there'' with coordinates sections, by working with the M-ellipsoid, and only for
the last step do we need to resort to general sections. We note that the above
conjecture is closely related to the Bourgain-Tzafriri restricted invertibility
principle\cite{bour-tza}, and indeed our proof for ellipsoids reduces to it.

\subsection{A Factorization Approach for Vector Balancing.}   

While Theorem~\ref{thm:tightness} gives an efficient and approximately
optimal method of balancing a given set of vectors, it does not give
an efficiently computable tight upper bound on the vector balancing constant
or on hereditary discrepancy. Even though we proved that, after an
appropriate scaling, the volume lower bound also gives an upper bound
on the vector balancing constant, we are not aware of an efficient
algorithm for computing the volume lower bound, which is itself a
maximum over an exponential number of terms. To address this
shortcoming, we study a different approach to vector balancing which
relies on applying Banaszczyk's theorem in an optimal
way in order to get an efficiently computable, and nearly tight, upper
bound on both vector balancing constants and hereditary discrepancy.

Recall that Banaszczyk's vector balancing theorem states that if a
body $K$ has Gaussian measure at least $1/2$, then $\vb(B_2^n, K) \le
5$. In order to apply the theorem to bodies $K$ of small Gaussian
measure, we can use rescaling.  In particular, if $r$ is the smallest
number such that the Gaussian measure of $rK$ is $\frac12$, then the
theorem tells us that $\vb(B_2^n, K) \le 5r$. A natural way to use
this upper bound for bodies $C$ different from $B_2^n$ is to find a
mapping of $C$ into $B_2^n$, and then use the theorem as above. As an
illustration of this idea, let us see how we can get nearly tight
bounds on $\vb(B^n_p, B^n_q)$ (the $\ell_p$ and $\ell_q$ balls) by
applying Banaszczyk's theorem. Let us take an arbitrary sequence of
points $u_1, \ldots, u_N \in B_p^n$, and rescale them to define new
points $v_i \eqdef {u_i}/{\max\{1, n^{1/2 -1/p}\}}$. The rescaled
points $v_1, \ldots, v_N$ lie in $B_2^n$ and we can apply Banaszczyk's
theorem to them and the convex body $K \eqdef L \sqrt{q} n^{1/q}
B_q^n$, which has Gaussian measure at least $\frac12$ as long as we
choose $L$ to be a large enough constant. We get that there exist
signs $\eps_1, \ldots, \eps_N \in \{-1, 1\}$ such that
\begin{multline*}
\left\|\sum_{i = 1}^N{\eps_i v_i}\right\|_K \le 5
\iff\\
\left\|\sum_{i = 1}^N{\eps_i u_i}\right\|_q \le
 5L\sqrt{q}\max\{n^{1/q}, n^{1/q + 1/2 -1/p}\}. 
\end{multline*}
In other words, we have that 
\[
\vb(B_p^n, B_q^n) \le
5L\sqrt{q}\max\{n^{1/q}, n^{1/q + 1/2 -1/p}\}.
\]
The volume lower bound (Lemmas~\ref{lem:vol-lb}) can be used to show
that this bound is tight up to the $O(\sqrt{q})$ factor. Indeed one
can show that $B_p^n$ contains $n$ vectors $u_1, \ldots u_n$ such that
the matrix $U \eqdef (u_1, \ldots, u_n)$ has determinant $\det(U) \ge
e^{-1} \max\{1, n^{1/2 -1/p}\}$ (see~\cite{Ball89}~or~\cite{N15}). By
standard estimates, $\vol(B_q^n)^{1/n} \ge c n^{1/q}$ for an absolute
constant $c >0$. Plugging these estimates into Lemma~\ref{lem:vol-lb}
shows $\vb(B_p, B_q) \ge c' \max\{n^{1/q}, n^{1/q + 1/2 -1/p}\}$ for a
constant $c' > 0$.

It is easy to see that, unlike the example above, in general simply
rescaling $C$ and $K$ and applying Banaszczyk's theorem to the
rescaled bodies may not give a tight bound on $\vb(C, K)$. However, we
will show that we can get such tight bounds if we expand the class of
transformations we allow on $C$ and $K$ from simple rescaling to
arbitrary linear transformations.  It turns out that the most
convenient language for this approach is that of linear operators
between normed spaces. We can generalize the notion of a vector
balancing constant between a pair of convex bodies to arbitrary linear
operators $U:X \to Y$ between two $n$-dimensional normed spaces $X$,
with norm $\| \cdot\|_X$, and $Y$, with norm $\| \cdot\|_Y$, as follows
\begin{multline}\label{eq:vb-oper}
\vb(U) = \sup%\Biggl\{
  \min_{\eps_1, \ldots, \eps_N \in \{-1, 1\}} 
  \biggl\|\sum_{i = 1}^N \eps_i U(x_i)\biggr\|_Y,%\Biggr\}
\end{multline}
where $B_X = \{x: \|x\|_X \le 1\}$ is the unit ball of $X$, and the
supremum is over positive integers $N$ and sequences $x_1, \ldots, x_N
\in B_X$.  This
definition is indeed a generalization of the geometric one. If $C$ and
$K$ are two centrally symmetric convex bodies in $\R^n$, and we define
the corresponding normed spaces $X_C = (\R^n, \|\cdot\|_C)$ and $X_K =
(\R^n, \|\cdot\|_K)$, then the vector balancing constant $\vb(I)$ of
the formal identity operator $I:X_C \to X_K$ recovers $\vb(C,
K)$. However, the more abstract setting makes it plain that a simple
rescaling is not the right approach to applying Banaszczyk's theorem
to arbitrary norms: if $X$ is an arbitrary norm, then $X$ and $B_2^n$
may not be defined on the same vector space, and rescaling $B_X$ so
that it is a subset of $B_2^n$ does not even make sense. Instead, when
dealing with general norms, it becomes very natural to embed $B_X$
into $B_2^n$ via a linear map $T:X \to \ell_2^n$
so that $T(B_X) \subseteq B_2^n$. Our approach is
based on this idea, and, in particular, on choosing such a map $T$
optimally.

To formalize the above, we use the $\ell$-norm, which has been
extensively studied in the theory of operator ideals, and in
asymptotic convex geometry (see
e.g.~\cite{TJ-book,Pisier-book,AGM-book}). For a linear operator
$S:\ell_2^n \to Y$ into an $n$-dimensional normed space $Y$ with norm
$\|\cdot\|_Y$, the $\ell$-norm of $S$ is defined as
\[
\ell(S) \eqdef \left( \int \|S(x)\|_Y^2 d\gamma_n(x) \right)^{1/2},
\]
where $\gamma_n$ is the standard Gaussian measure on $\R^n$. I.e., if
$Z$ is a standard Gaussian random variable in $\R^n$, then $\ell(S) =
(\E \|S(Z)\|_Y^2)^{1/2}$. It is direct to verify that $\ell(\cdot)$ is a
norm on the space of linear operators from $\ell_2^n$ to $Y$, for any
normed space $Y$ as above. The reason the $\ell$-norm is useful to us
is the fact that the smallest $r$ for which the set $K = \{x \in \R^n:
\|Sx\|_Y \le r\}$  has Gaussian measure at least $1/2$ is approximately
$\ell(S)$, due to the  concentration of measure phenomenon. 

We now define our main tool: a factorization constant $\lambda$, which, for any
two $n$-dimensional normed spaces $X$ and $Y$ and an operator $U:X \to
Y$ is defined by
\[
\lambda(U) \eqdef \inf \{\ell(S)\|T\|: T: X \to \ell_2^n,\ S: \ell_2^n
\to Y, U = ST\}.
\]
In other words, $\lambda(U)$ is the minimum of $\ell(S)\|T\|$ over all
ways to factor $U$ through $\ell_2^n$ as $U = ST$. Here $\|T\|$ is the
operator norm, equal to $\max\{{\|Tx\|_2}/{\|x\|_X}\}$. 
This definition
captures an optimal application of Banaszczyk's
theorem. Using the theorem, it is not hard to show that $\vb(U) \le
C\lambda(U)$ for an absolute constant $C$. Our main result is showing
$\vb(U)$ and $\lambda(U)$ are in fact equal up to a factor which is
polynomial in $\log n$. To prove this, we formulate $\lambda(U)$ as a
convex minimization problem. Such a formulation is important both for
our structural results, which rely on Lagrange duality, and also for
giving an algorithm to compute $\lambda(U)$ efficiently, and,
therefore, approximate $\vb(U)$ efficiently, which turns out to be
sufficient to approximate hereditary discrepancy in arbitrary norms.

The most immediate way to formulate $\lambda(U)$ as an optimization
problem is to minimize $\ell(UT^{-1})$ over operators $T:X \to
\ell_2^n$ and subject to the constraint $\|T\|\le 1$.
Unfortunately, this optimization problem is not convex in $T$: the
value of the objective function is finite for any nonzero $T$, but
infinite for $0 = \frac12(T + (-T))$, for example. The key observation
that allows us to circumvent this issue is that the objective function
is completely determined by the operator $A \eqdef T^*T$, and is in
fact convex in $A$. Here $T^*$ is the dual operator of $T$. We use $f(A)$ to
denote this objective function, i.e.~to denote $\ell(UT^{-1})$ where
$T$ is an operator such that $T^*T = A$. In the full version we prove
that this function is well-defined and convex. Then, our convex formulation of
$\lambda(U)$ is
\begin{align*}
  &\inf  f(A) \\
  &\text{s.t.}\notag\\
  &A: X \to X^*,\|A\| \le 1\\
  &A \succ 0.
\end{align*}
Above, $X^*$ is the dual space of $X$, and $\|A\|$ is the operator
norm. The first constraint is equivalent to the constraint $\|T\|\le
1$ where $U = ST$ is the factorization in the definition of
$\lambda(U)$. The last constraint says that $A$ should be positive
definite, which is important so that $A$ can be written as $T^*T$ and
$f(A)$ is well-defined. 

 We utilize this convex formulation
and Lagrange duality to derive a
dual formulation of $\lambda(U)$ as a supremum over ``dual
certificates''. Such a formulation is useful in approximately
characterizing $\vb(U)$ in terms of $\lambda(U)$ because it reduces
our task to relating the dual certificates to the terms in
the volume lower bound~\eqref{eq:vol-lb}. If we can show that every
dual certificate bounds from below one of the terms of the volume
lower bound (up to factors polynomial in $\log n$), then we can
conclude that $\lambda(U)$ also bounds the volume lower bound from
below, and therefore $\vb(U)$ as well. 

Before we can give the dual formulation, we need to introduce the
dual norm $\ell^*$ of the $\ell$-norm, defined via trace duality: for
any linear opartor $R: Y \to \ell_2^n$, let
\[
\ell^*(R) \eqdef \sup\{\tr(RS): S: \ell_2^n \to Y, \ell(S) \le 1\}.
\]
The norms $\ell$ and $\ell^*$ form a dual pair, and in particular we have
\[
\ell(S) = \sup\{\tr(RS): R:Y\to\ell_2^n, \ell^*(R) \le 1\}.
\]
For a finite dimensional space $Y$, both suprema above are achieved.

The derivation of our dual formulation uses standard tools, but is
quite technical due to the complicated nature of the function
$f(A)$. We give the formulation for norms $X$ such that $B_X =
\conv\{\pm x_1, \ldots, \pm x_m\}$. This is without loss of generality
since every symmetric convex body can be approximated by a symmetric
polytope (but has implications for the complexity of our algorithms). The dual
formulation is as follows:
\begin{align*}
  &\sup \tr((RU(\sum_{i = 1}^m{p_i x_i \otimes  x_i})U^*R^*)^{1/3})^{3/2}\\
  \text{s.t.}\ \ \ 
  &R: Y \to \ell_2^n, \ell^*(R) \le 1 \\
  &\sum_{i = 1}^m{p_i} = 1, \ \ 
  p_1, \ldots, p_m \ge 0. 
\end{align*}
Above $x_i \otimes x_i$ is the rank-1 operator from the dual space
$X^*$ to $X$, given by $(x_i \otimes x_i)(x^*) = \langle x^*,
x_i\rangle x_i$. 

We relate  the volume lower bound to this dual via deep inequalities
between the $\ell^*$ and the $\ell$ norms ($K$-convexity), and between the
$\ell$ norm and packing and covering numbers (Sudakov's
minoration). Our second main result is the theorem below.

\begin{theorem}\label{thm:fact-main}
  There exists a constant $C$ such that for any two $n$-dimensional
  normed spaces $X$ and $Y$, and any linear operator $U:X \to Y$
  between them, we have
  \[
  \frac1C \le \frac{\lambda(U)}{\vb(U)} \le C (1 + \log n)^{5/2}.
  \]
  Moreover, for any vectors $u_1, \ldots, u_N$ and convex body $K$ in
  $\R^n$ we can define a norm $X$ on $\R^n$ so that for the space $Y$
  with unit ball $K$ and the identity map $I:X \to Y$,
  \[
  \frac{\lambda(I)}{C(1 + \log n)^{5/2}} 
  \le \hd((u_i)_{i = 1}^N, K) \le \vb(I) 
  \le C \lambda(I).
  \]
  Finally, $\lambda(U)$ is computable in polynomial time given
  appropriate access to $X$ and $Y$.
  \footnote{See the full version of the paper for the necessary assumptions.}
\end{theorem}

\section*{Acknowledgment}

Daniel Dadush is supported by NWO Veni grant 639.071.510. Aleksandar
Nikolov is supported by an NSERC Discovery Grant (RGPIN-2016-06333). 

The authors would like to thank the American Institute of Mathematics
for hosting a workshop on Hereditary Discrepancy and Factorization
Norms, where work on this project began.

% trigger a \newpage just before the given reference
% number - used to balance the columns on the last page
% adjust value as needed - may need to be readjusted if
% the document is modified later
%\IEEEtriggeratref{8}
% The "triggered" command can be changed if desired:
%\IEEEtriggercmd{\enlargethispage{-5in}}

% references section

% can use a bibliography generated by BibTeX as a .bbl file
% BibTeX documentation can be easily obtained at:
% http://www.ctan.org/tex-archive/biblio/bibtex/contrib/doc/
% The IEEEtran BibTeX style support page is at:
% http://www.michaelshell.org/tex/ieeetran/bibtex/
\bibliographystyle{IEEEtran}
% argument is your BibTeX string definitions and bibliography database(s)
\bibliography{../Discrepancy.bib}
%
% <OR> manually copy in the resultant .bbl file
% set second argument of \begin to the number of references
% (used to reserve space for the reference number labels box)




% that's all folks
\end{document}


