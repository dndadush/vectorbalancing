\section{Approximate Caratheodory Estimates}
\label{sect:apx-carath}

In this section we introduce connections between vector balancing, and
approximate versions of Caratheodory's theorem. Recall that, by
Caratheodory's theorem, for any $N$ vectors $u_1, \ldots, u_N \in
\R^n$ and a point $z$ in their convex hull, there exist $k \le n+1$
vectors $v_1, \ldots, v_k \in \{u_1, \ldots, u_N\}$ such that $z \in
\conv\{v_1, \ldots, v_k\}$. While this is easily seen to be optimal in
general, improvements are possible if we only need to approximately
represent $z$. For example, it is well-known that if $u_1, \ldots, u_N
\in B_2^n$, then for any $z \in \conv\{u_1, \ldots, u_N\}$ and any
$\eps \in (0,1)$,  there exist $k = O(\eps^{-2})$ vectors $v_1,
\ldots, v_k \in \{u_1, \ldots, u_N\}$ such that $\left\|z - \frac1k
  \sum_{i = 1}^k{v_i}\right\|_2 \le \eps$. Note that the bound on $k$
in this approximate version of Caratheodory's theorem is independent
of the dimension $n$, and that $z$ is close to the avarage of $v_1,
\ldots, v_k$, rather than merely in their convex hull. This
dimension-independent bound has been used to speed up algorithms in
computational geometry...

\snote{Dig up some refs and applications. Reference for the $\ell_2$
  approximate Caratheodory theorem?}

We call bounds of the type given above for vectors in $B_2^n$
approximate Caratheodory estimates. Such estimates are also known for
$B_p^n$, where the distance from $z$ is measured in the $\ell_p^n$
norm, and follow from type and cotype theory. (Other proofs are also
known.) Here we systematically explore approximate Caratheodory
estimates and connect them to vector balancing. Let us define the
$k$-vector approximate Caratheodory constant $\ac_k(C,K)$ from a
convex body $C \subset \R^n$ to a symmetric convex body $K\subset
\R^n$ as the smallest constant $a$ such that for any $u_1, \ldots, u_N
\in C$, and any $z \in \conv\{u_1, \ldots, u_N\}$, there exist (not
necessarily distinct) vectors $v_1, \ldots, v_k \in \{u_1, \ldots,
u_N\}$ such that
\begin{equation}\label{eq:ac-defn}
\left\|z - \frac1k \sum_{i = 1}^k{v_i}\right\|_K \le \frac{a}{k}.
\end{equation}
We define $\ac(C,K) = \sup\{\ac_k(C, K): k \in \N\}$.

Our first result shows that the approximate Caratheodory constant is
bounded in terms of hereditary discrepancy, and, therefore, in terms
of the vector balancing constant.  Since our proof is algorithmic,
together with Theorem~\ref{thm:tightness} it also gives us a
polynomial time algorithm to compute the vectors $v_1, \ldots, v_k$
whose average is close to $z$.

\begin{theorem}\label{thm:ac-all-k}
  For any convex body $C \subset \R^n$, and any symmetric convex body
  $K \subset \R^n$, we have
  \[
  \ac(C,K) \le \vb(C-C,K).
  \]
  Moreover, there exists a randomized polynomial time algorithm that,
  given $u_1, \ldots, u_N \in C$, and $z$ in their convex hull,
  computes vectors $v_1, \ldots, v_k \in \{u_1, \ldots, u_N\}$ such
  that \eqref{eq:ac-defn} holds with $a = O(\log n \vb(C-C, K)$.
\end{theorem}

Our proof of Theorem~\ref{thm:ac-all-k} is based on the following
lemma, which is a slight extension of a result by Lov\'asz, Spencer,
and Vesztergombi~\cite{LSV}. 

\begin{lemma}\label{lm:ac-to-herdisc}
  For any symmetric convex body $K \subset\R^n$, any $u_1, \ldots, u_N
  \in \R^n$, and any $w \in [0,1]^N$, there exists an $x\in \{0,1\}^N$
  such that $\sum_{i = 1}^N{x_i} \le \sum_{i = 1}^N{w_i}$, and
  \[
  \left\|\sum_{i = 1}^n(w_i - x_i)u_i\right\|_K \le \hd((u_i)_{i =  1}^N, K).
  \]
\end{lemma}
\begin{proof}
  We first show that the lemma holds for every $w \in \{0, \frac12,
  1\}^N$ with a better constant. Specifically, we show that for any $w
  \in \{0, \frac12, 1\}^N$, there exists an $x \in \{0,1\}^N$ such that
  \[
  \left\|\sum_{i = 1}^n(w_i - x_i)u_i\right\|_K \le  \frac12 \hd((u_i)_{i =  1}^N, K),
  \]
  and $\sum_{i = 1}^N x_i \le \sum_{i = 1}^N{w_i}$. Towards this goal,
  let us define $S = \{i \in [n]: w_i = \frac12\}$. By the definition
  of hereditary discrepancy, there exists a vector of signs
  $(\eps_i)_{i \in S} \in \{-1,+1\}^S$ such that 
  \[
  \left\|\sum_{i \in S}{\eps_i u_i}\right\|_K   \le \hd((u_i)_{i = 1}^N,  K).
  \] 
  Let us define $x' \in \{-1, 0,   1\}^N$ by 
  \[
  x'_i =
  \begin{cases}
    \eps_i &i \in S\\
    0 &i \not \in S
  \end{cases}.
  \]
  Observe, first, that $w+\frac12 x'$ and $w-\frac12 x'$ both belong to $\{0,1\}^N$,
  and, second, that 
  \[
  \min\left\{\sum_{i = 1}^N{w_i+\frac12 x'_i}, \sum_{i = 1}^N{w_i-\frac12 x'_i}\right\}
  \le \sum_{i = 1}^N{w_i}.
  \]
  We can then take the vector $x \in \{w + \frac12 x', w-\frac12 x'\}$ that achieves the minimum
  on the right hand side above, and we have
  \[
  \left\|\sum_{i = 1}^n(w_i - x_i)u_i\right\|_K 
  =\frac12  \left\|\sum_{i \in S}{\eps_i u_i}\right\|_K 
  \le  \frac12 \hd((u_i)_{i =  1}^N, K),
  \]
  as desired.

  To finish the proof of the lemma, it suffices to prove it for $w$
  such that $w_i$ has a finite binary expansion for each $i$,
  i.e.~$w_i = b_12^{-1} + \ldots + b_k2^{-k}$ for some finite $k$ and
  $b_1, \ldots, b_k \in \{0,1\}$. We do so by induction. Let us assume
  that the lemma is proved for all $w \in 2^{-k}\mathbb{Z}^N\cap
  [0,1)^N$ for some $k \ge 1$. (Note that the case $k=1$ was already
  proved above.) We will show that it then also holds for every $w \in
  2^{-k-1}\mathbb{Z}^N\cap [0,1)^N$. Fix such a $w$, and write it as
  $w = w' + \frac12 w''$, where $w' \in \{0, \frac12\}^N$ and $w'' \in
  2^{-k}\mathbb{Z}^N\cap [0,1)^N$. By the induction hypothesis,
  there exists an $x'' \in \{0,1\}^N$ such that 
  \[
  \left\|\sum_{i = 1}^n(w_i'' - x_i'')u_i\right\|_K \le  \hd((u_i)_{i =  1}^N, K),
  \]
  and $\sum_{i = 1}^N x''_i \le \sum_{i = 1}^N w''_i$. We have $w' +
  \frac12 x'' \in \{0,\frac12, 1\}$, and, as we have already shown above, there
  exists an $x \in \{0,1\}^N$ such that 
  \begin{equation}\label{eq:transf-ind}
  \left\|\sum_{i = 1}^n(w_i' + \frac12 x''_i - x_i)u_i\right\|_K 
  \le \frac12 \hd((u_i)_{i =  1}^N, K),
  \end{equation}
  and 
  \[
  \sum_{i = 1}^N x_i \le \sum_{i = 1}^N{w'_i} + \frac12 \sum_{i = 1}^N{x''_i}
  \le \sum_{i = 1}^N{w'_i} + \frac12 \sum_{i = 1}^N{w''_i} 
  =   \sum_{i = 1}^N w_i.
  \]
  The inequality \eqref{eq:transf-ind} and the triangle inequality then imply
  \begin{align*}
  \left\|\sum_{i = 1}^n(w_i - x_i)u_i\right\|_K 
  &= 
  \left\|\sum_{i = 1}^n(w_i' + \frac12 w''_i - x_i)u_i\right\|_K \\
  &\le 
  \left\|\sum_{i = 1}^n(w_i' + \frac12 x''_i - x_i)u_i\right\|_K 
  + 
  \frac12 \left\|\sum_{i = 1}^n(w_i'' - x_i'')u_i\right\|_K 
  \le  \hd((u_i)_{i =  1}^N, K).
  \end{align*}
  This completes the proof of the lemma. 
\end{proof}

\snote{This is also used in the proof of the volume lower bound (where
  the condition on $\sum x_i$ is not needed). Move it there?}

\begin{proof}[Proof of Theorem~\ref{thm:ac-all-k}]
  Let us fix some $k \in \N$, vectors $u_0, u_1, \ldots, u_N$ and $z
  \in \conv\{u_0, \ldots, u_N\}$. (We index the vectors from $0$ for
  notational convenience.) Define the new vectors $u'_1, \ldots u'_N$
  by $u'_i = u_i - u_0$. We have that $z - u_0 \in \conv\{u_1',
  \ldots, u_N'\}$, and we can write $z - u_0 = \sum_{i =
    1}^N{\lambda_i u'_i}$ for some $\lambda_1, \ldots, \lambda_N \ge
  0$ such that $\sum_{i = 1}^N{\lambda_i} = 1$. Let $w$ be the fractional
  part of the vector $k\lambda$, i.e.~$w_i = k\lambda_i - \lfloor
  k\lambda_i \rfloor$. By Lemma~\ref{lm:ac-to-herdisc}, there exists a
  vector $x \in \{0,1\}^N$ such that  $\sum_{i = 1}^N{x_i} \le \sum_{i = 1}^N{w_i}$, and
  \[
  \left\|\sum_{i = 1}^n(w_i - x_i)u'_i\right\|_K \le \hd((u'_i)_{i =  1}^N, K).
  \]
  Let us define vectors $v'_1, \ldots, v'_\ell \in \{u'_1, \ldots,
  u'_N\}$ by taking $\lfloor  k\lambda_i\rfloor + x_i$ copies of
  $u'_i$. Clearly,
  \[
  \ell = \sum_{i = 1}^N{\lfloor k\lambda_i\rfloor + x_i} \le \sum_{i
    = 1}^N{\lfloor k\lambda_i\rfloor + w_i}  = k\sum_{i =
    1}^N{\lambda_i} = k.
  \]
  We then define $v_1, \ldots, v_k \in \{u_0, \ldots, u_N\}$ by taking
  $k-\ell$ copies of $u_0$, and also taking the vectors $v'_i + u_0$
  for every $i
  \in [\ell]$.  Observe that
  \begin{align*}
    \left\|z - \frac1k \sum_{i = 1}^k{v_i}\right\|_K
    &= 
    \left\|z -u_0 - \frac1k \sum_{i = 1}^\ell{v'_i}\right\|_K\\
    &= \frac{1}{k} \left\|\sum_{i = 1}^n(w_i - x_i)u'_i\right\|_K
    \le \hd((u'_i)_{i = 1}^N, K) \le \vb(C - C, K).
  \end{align*}
  This completes the proof. 
\end{proof}