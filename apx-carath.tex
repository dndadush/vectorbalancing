\section{Approximate Caratheodory Estimates}
\label{sect:apx-carath}

In this section we introduce connections between vector balancing, and
approximate versions of Caratheodory's theorem. Recall that, by
Caratheodory's theorem, for any $N$ vectors $u_1, \ldots, u_N \in
\R^n$ and a point $z$ in their convex hull, there exist $k \le n+1$
vectors $v_1, \ldots, v_k \in \{u_1, \ldots, u_N\}$ such that $z \in
\conv\{v_1, \ldots, v_k\}$. While this is easily seen to be optimal in
general, improvements are possible if we only need to approximately
represent $z$. For example, it is well-known that if $u_1, \ldots, u_N
\in B_2^n$, then for any $z \in \conv\{u_1, \ldots, u_N\}$ and any
$\eps \in (0,1)$,  there exist $k = O(\eps^{-2})$ vectors $v_1,
\ldots, v_k \in \{u_1, \ldots, u_N\}$ such that $\left\|z - \frac1k
  \sum_{i = 1}^k{v_i}\right\|_2 \le \eps$. Note that the bound on $k$
in this approximate version of Caratheodory's theorem is independent
of the dimension $n$, and that $z$ is close to the avarage of $v_1,
\ldots, v_k$, rather than merely in their convex hull. This
dimension-independent bound has been used to speed up algorithms in
computational geometry...

\snote{Dig up some refs and applications. Reference for the $\ell_2$
  approximate Caratheodory theorem?}

We call bounds of the type given above for vectors in $B_2^n$
approximate Caratheodory estimates. Such estimates are also known for
$B_p^n$, where the distance from $z$ is measured in the $\ell_p^n$
norm, and follow from type and cotype theory. (Other proofs are also
known.) Here we systematically explore approximate Caratheodory
estimates and connect them to vector balancing. Let us define the
$k$-vector approximate Caratheodory constant $\ac_k(C,K)$ from a
convex body $C \subset \R^n$ to a symmetric convex body $K\subset
\R^n$ as the smallest constant $a$ such that for any $u_1, \ldots, u_N
\in C$, and any $z \in \conv\{u_1, \ldots, u_N\}$, there exist (not
necessarily distinct) vectors $v_1, \ldots, v_k \in \{u_1, \ldots,
u_N\}$ such that
\begin{equation}\label{eq:ac-defn}
\left\|z - \frac1k \sum_{i = 1}^k{v_i}\right\|_K \le \frac{a}{k}.
\end{equation}
We define $\ac(C,K) = \sup\{\ac_k(C, K): k \in \N\}$.

Our first result shows that the approximate Caratheodory constant is
bounded in terms of hereditary discrepancy, and, therefore, in terms
of the vector balancing constant.  Since our proof is algorithmic,
together with Theorem~\ref{thm:tightness} it also gives us a
polynomial time algorithm to compute the vectors $v_1, \ldots, v_k$
whose average is close to $z$.

\begin{theorem}\label{thm:ac-all-k}
  For any convex body $C \subset \R^n$, and any symmetric convex body
  $K \subset \R^n$, we have
  \[
  \ac(C,K) \le \vb(C-C,K).
  \]
  Moreover, there exists a randomized polynomial time algorithm that,
  given $u_1, \ldots, u_N \in C$, and $z$ in their convex hull,
  computes vectors $v_1, \ldots, v_k \in \{u_1, \ldots, u_N\}$ such
  that \eqref{eq:ac-defn} holds with $a = O(\log n \vb(C-C, K)$.
\end{theorem}

Our proof of Theorem~\ref{thm:ac-all-k} is based on the following
lemma, which is a slight extension of a result by Lov\'asz, Spencer,
and Vesztergombi~\cite{LSV}.

\begin{lemma}\label{lm:ac-to-herdisc}
  For any symmetric convex body $K \subset\R^n$, any $u_1, \ldots, u_N
  \in \R^n$, and any $w \in [0,1]^N$, there exists an $x\in \{0,1\}^N$
  such that $\sum_{i = 1}^N{x_i} \le \sum_{i = 1}^N{w_i}$, and
  \[
  \left\|\sum_{i = 1}^n(w_i - x_i)u_i\right\|_K \le \hd((u_i)_{i =  1}^N, K).
  \]
\end{lemma}